
\section{The \textit{teddf} command}\label{sec_teddf}
{\tt teddf} estimates directional distance function with undesirable outputs for technical efficiency measurement.

\subsection{Syntax}
\begin{stsyntax}
	teddf\
	\textit{X\varlist} = \textit{Y\varlist}:\textit{B\varlist} \optif \optin,\
	\underbar{d}mu(\varname)
	\optional{
		    \underbar{t}ime(\varname)
		    gx(\varlist)
		    gy(\varlist)
		    gb(\varlist)
		    \underbar{nonr}adial
		    wmat(name)
		    vrs
		    rf(\varname)
		    \underbar{win}dow(\num)
		    \underbar{bi}ennial 
		    \underbar{seq}uential
		    \underbar{glo}bal
		    brep(\num)
		    alpha(real)
		    tol(real)
		    maxiter(\num)
		    \underbar{sav}ing(filename[,replace])
		    frame(framename)
		    nodots
		    noprint
		    \underbar{noch}eck
 }
\end{stsyntax}


\subsection{Options}

\hangpara
{\tt dmu(\varname)} specifies names of DMUs. It is required.

\hangpara
{\tt time(\varname)} specifies the time variable for panel data.

\hangpara
{\tt gx(\varlist)} specifies direction components for input adjustment. The order of variables specified in gx() should as the same in \textit{X\varlist}. By default, gx takes the opposite of \textit{X\varlist}.

\hangpara
{\tt gy(\varlist)} specifies direction components for desirable output adjustment. The order of variables specified in gy() should as the same in \textit{Y\varlist}. By default, gy takes \textit{Y\varlist}.

\hangpara
{\tt gb(\varlist)} specifies direction components for undesirable output adjustment. The order of variables specified in gb() should as the same in \textit{B\varlist}. By default, gb takes the opposite of \textit{B\varlist}.

\hangpara
{\tt nonradial} specifies using the nonradial directional distance measure.

\hangpara
{\tt wmat(name)} specifies a weight matrix for adjustment of input and output variable for the nonradial directional distance measure. The default is wmatrix= (1,...,1).

\hangpara
{\tt vrs} specifies production technology with variable returns to scale. By default, production technology with constant returns to scale is assumed.

\hangpara
{\tt rf(\varname)} specifies the indicator variable that defines which data points of outputs and inputs form the technology reference set.

\hangpara
{\tt window(\num)}  specifies using window production technology with the \num-period bandwidth.

\hangpara
{\tt biennial}  specifies using biennial production technology.

\hangpara
{\tt sequential} specifies using sequential production technology.

\hangpara
{\tt global} specifies using global production technology.

\hangpara
{\tt brep(\num)} specifies the number of bootstrap replications. The default is brep(0) specifying performing the estimator without bootstrap. Typically, it requires 1,000 or more replications for bootstrap DEA methods.

\hangpara
{\tt alpha(real) } sets the size of the subsample bootstrap. By default, alpha(0.7) indicates subsampling $N^{0.7}$ observations out of the $N$ original reference observations.

\hangpara
{\tt tol(real)} specifies the convergence-criterion tolerance  for LinearProgram(). The default value of tol is 1e-8.

\hangpara
{\tt maxiter(\num) } specifies the maximum number of iterations for LinearProgram(). The default value of maxiter is 16000.

\hangpara
 {\tt saving(filename[,replace])}  specifies a filename to store the results.
 
\hangpara
 {\tt frame(name)} specifies a framename to stroe the results.
 
\hangpara
{\tt nodots} suppress iteration dots.

\hangpara
{\tt noprint} suppress suppress display of the results.

\hangpara
{\tt nocheck} suppress checking for new version. It is suggested to be used for saving time when internet connection is unavailable.


\section{The \textit{gtfpch} command}\label{sec_gtfpch}
{\tt gtfpch} measures total factor productivity change with undesirable outputs using Malmquist–Luenberger productivity index or Luenberger indicator.

\subsection{Syntax}
\begin{stsyntax}
	gtfpch\
	\textit{X\varlist} = \textit{Y\varlist}:\textit{B\varlist} \optif \optin,\
	\optional{
		\underbar{d}mu(\varname)
		\underbar{luen}berger
		ort(\ststring)
		gx(\varlist)
		gy(\varlist)
		gb(\varlist)
		\underbar{nonr}adial
		wmat(name)
		\underbar{win}dow(\num)
		\underbar{bi}ennial 
		\underbar{seq}uential
		\underbar{glo}bal	
		fgnz
		rd	
		tol(real)
		maxiter(\num)
		\underbar{sav}ing(filename[,replace])
		frame(framename)
		noprint
		\underbar{noch}eck		
	}
\end{stsyntax}


\subsection{Options}

\hangpara
{\tt dmu(\varname)} specifies names of DMUs. 

\hangpara
{\tt luenberger} specifies estimating Luenberger productivity indicator. The default is Malmquist–Luenberger productivity index based on the radial directional distance function.

\hangpara
{\tt ort(\ststring)} specifies the oriention. The default is ort(input), meaning the input oriented productivity index/indicator. ort(output) means the output oriented productivity index/indicator. ort(hybrid) means the hybrid-direction productivity index/indicator.

\hangpara
{\tt gx(\varlist)} specifies direction components for input adjustment. The order of variables specified in gx() should as the same in \textit{X\varlist}. By default, gx=(0,..,0) for ort(output); gx=-\textit{X\varlist} for ort(input) and ort(hybrid).

\hangpara
{\tt gy(\varlist)} specifies direction components for desirable output adjustment. The order of variables specified in gy() should as the same in \textit{Y\varlist}. By default, gy=(0,..,0) for ort(input); gy=\textit{Y\varlist} for ort(output) and ort(hybrid).

\hangpara
{\tt gb(\varlist)} specifies direction components for undesirable output adjustment. The order of variables specified in gb() should as the same in \textit{B\varlist}. By default, gb=(0,..,0) for ort(input); gb=-\textit{B\varlist} for ort(output) and ort(hybrid).

\hangpara
{\tt nonradial} specifies using the non-radial directional distance measure.

\hangpara
{\tt wmat(name)} specifies a weight matrix for adjustment of input and output variable for the nonradial directional distance measure.

\hangpara
{\tt window(\num)}  specifies using window production technology with the \num-period bandwidth.

\hangpara
{\tt biennial}  specifies using biennial production technology.

\hangpara
{\tt sequential} specifies using sequential production technology.

\hangpara
{\tt global} specifies using global production technology.

\hangpara
{\tt fgnz} specifies specifies decomposing TFP change following the spirit of \cite{Fare1994} method.

\hangpara
{\tt rd} specifies decomposing TFP change following the spirit of \cite{Ray1997} method.

\hangpara
{\tt tol(real)} specifies the convergence-criterion tolerance  for LinearProgram(). The default value of tol is 1e-8.

\hangpara
{\tt maxiter(\num) } specifies the maximum number of iterations for LinearProgram(). The default value of maxiter is 16000.

\hangpara
{\tt saving(filename[,replace])}  specifies a file name to store the results.

\hangpara
{\tt frame(name)}  specifies a frame name to store the results.

\hangpara
{\tt noprint}   suppress suppress display of the results.

\hangpara
{\tt nocheck} suppress checking for new version. It is suggested to be used for saving time when internet connection is unavailable.

\section{Example}\label{sec_example}
To exemplify the use of the commands described above, we use an input-output data set of China's provinces for the period of 2013-2015 which is obtained from a recent publication, \cite{YAN2020}. The dataset includes three input variables (capital, labor, and energy), one desirable output (real GDP), and one undesirable output ($CO_2$ emissions). The data are described as follows.

\begin{stlog}
	. 
. use example.dta
{\smallskip}
. describe 
{\smallskip}
Contains data from example.dta
  obs:            90                          
 vars:             7                          6 Aug 2020 12:12
\HLI{135}
              storage   display    value
variable name   type    format     label      variable label
\HLI{135}
Province        str12   \%12s                  province name
year            int     \%10.0g                year
K               float   \%9.0g                 capital stock (in 100 million 1997 CNY)
L               double  \%10.0g                employment (in 10 thousand persons)
E               double  \%10.0g                energy consumption (in million tons of standard coal)
Y               float   \%9.0g                 real GDP (in 100 million 1997 CNY)
CO2             float   \%15.1f                carbon dioxide emission (in kg)
\HLI{135}
Sorted by: 
{\smallskip}
. 

\end{stlog}


\subsection{Application of \textit{teddf}}
The estimation of the directional distance function model proposed by \cite{Chung1997} as follows. The corresponding results are displayed below the executed command. The Dv variable stores the values of the directional distance function of the DMUs. Note that the sav(ex.teddf.result) option saves the results in a new data file named ex.teddf.result.dta.

\begin{stlog}
	. 
. teddf K L= Y: CO2, dmu(Province) time(year) sav(ex.teddf.result,replace)
{\smallskip}
 The directional vector is (-K -L Y -CO2)
{\smallskip}
{\smallskip}
  Directional Distance Function Results:
    (Row: Row \# in the original data; Dv: Estimated value of  DDF.)
{\smallskip}
     {\TLC}\HLI{37}{\TRC}
     {\VBAR} Row       Province   year        Dv {\VBAR}
     {\LFTT}\HLI{37}{\RGTT}
  1. {\VBAR}   1          Anhui   2013    0.2917 {\VBAR}
  2. {\VBAR}   2          Anhui   2014    0.3589 {\VBAR}
  3. {\VBAR}   3          Anhui   2015    0.3735 {\VBAR}
  4. {\VBAR}   4        Beijing   2013   -0.0000 {\VBAR}
  5. {\VBAR}   5        Beijing   2014   -0.0000 {\VBAR}
  6. {\VBAR}   6        Beijing   2015   -0.0000 {\VBAR}
  7. {\VBAR}  88      Chongqing   2013    0.2068 {\VBAR}
  8. {\VBAR}  89      Chongqing   2014    0.2362 {\VBAR}
  9. {\VBAR}  90      Chongqing   2015    0.2570 {\VBAR}
 10. {\VBAR}   7         Fujian   2013    0.0877 {\VBAR}
 11. {\VBAR}   8         Fujian   2014    0.1423 {\VBAR}
 12. {\VBAR}   9         Fujian   2015    0.1482 {\VBAR}
 13. {\VBAR}  10          Gansu   2013    0.2894 {\VBAR}
 14. {\VBAR}  11          Gansu   2014    0.3679 {\VBAR}
 15. {\VBAR}  12          Gansu   2015    0.4425 {\VBAR}
 16. {\VBAR}  13      Guangdong   2013   -0.0000 {\VBAR}
 17. {\VBAR}  14      Guangdong   2014    0.0372 {\VBAR}
 18. {\VBAR}  15      Guangdong   2015    0.0487 {\VBAR}
 19. {\VBAR}  16        Guangxi   2013    0.2495 {\VBAR}
 20. {\VBAR}  17        Guangxi   2014    0.2751 {\VBAR}
 21. {\VBAR}  18        Guangxi   2015    0.2877 {\VBAR}
 22. {\VBAR}  19        Guizhou   2013    0.2795 {\VBAR}
 23. {\VBAR}  20        Guizhou   2014    0.3660 {\VBAR}
 24. {\VBAR}  21        Guizhou   2015    0.4460 {\VBAR}
 25. {\VBAR}  22         Hainan   2013    0.1920 {\VBAR}
 26. {\VBAR}  23         Hainan   2014    0.2533 {\VBAR}
 27. {\VBAR}  24         Hainan   2015    0.3076 {\VBAR}
 28. {\VBAR}  25          Hebei   2013    0.2237 {\VBAR}
 29. {\VBAR}  26          Hebei   2014    0.2913 {\VBAR}
 30. {\VBAR}  27          Hebei   2015    0.3486 {\VBAR}
 31. {\VBAR}  31   Heilongjiang   2013    0.1191 {\VBAR}
 32. {\VBAR}  32   Heilongjiang   2014    0.1401 {\VBAR}
 33. {\VBAR}  33   Heilongjiang   2015    0.1579 {\VBAR}
 34. {\VBAR}  28          Henan   2013    0.3024 {\VBAR}
 35. {\VBAR}  29          Henan   2014    0.3473 {\VBAR}
 36. {\VBAR}  30          Henan   2015    0.3597 {\VBAR}
 37. {\VBAR}  34          Hubei   2013    0.1463 {\VBAR}
 38. {\VBAR}  35          Hubei   2014    0.1870 {\VBAR}
 39. {\VBAR}  36          Hubei   2015    0.2051 {\VBAR}
 40. {\VBAR}  37          Hunan   2013    0.1579 {\VBAR}
 41. {\VBAR}  38          Hunan   2014    0.1891 {\VBAR}
 42. {\VBAR}  39          Hunan   2015    0.2286 {\VBAR}
 43. {\VBAR}  43        Jiangsu   2013    0.1451 {\VBAR}
 44. {\VBAR}  44        Jiangsu   2014    0.1613 {\VBAR}
 45. {\VBAR}  45        Jiangsu   2015    0.1549 {\VBAR}
 46. {\VBAR}  46        Jiangxi   2013    0.2358 {\VBAR}
 47. {\VBAR}  47        Jiangxi   2014    0.2748 {\VBAR}
 48. {\VBAR}  48        Jiangxi   2015    0.3122 {\VBAR}
 49. {\VBAR}  40          Jilin   2013    0.3361 {\VBAR}
 50. {\VBAR}  41          Jilin   2014    0.3433 {\VBAR}
 51. {\VBAR}  42          Jilin   2015    0.3663 {\VBAR}
 52. {\VBAR}  49       Liaoning   2013    0.1794 {\VBAR}
 53. {\VBAR}  50       Liaoning   2014    0.1832 {\VBAR}
 54. {\VBAR}  51       Liaoning   2015    0.1711 {\VBAR}
 55. {\VBAR}  52      Neimenggu   2013   -0.0000 {\VBAR}
 56. {\VBAR}  53      Neimenggu   2014   -0.0000 {\VBAR}
 57. {\VBAR}  54      Neimenggu   2015    0.0000 {\VBAR}
 58. {\VBAR}  55        Ningxia   2013   -0.0000 {\VBAR}
 59. {\VBAR}  56        Ningxia   2014    0.0000 {\VBAR}
 60. {\VBAR}  57        Ningxia   2015   -0.0000 {\VBAR}
 61. {\VBAR}  58        Qinghai   2013    0.4524 {\VBAR}
 62. {\VBAR}  59        Qinghai   2014    0.4928 {\VBAR}
 63. {\VBAR}  60        Qinghai   2015    0.5074 {\VBAR}
 64. {\VBAR}  67        Shaanxi   2013    0.4054 {\VBAR}
 65. {\VBAR}  68        Shaanxi   2014    0.4547 {\VBAR}
 66. {\VBAR}  69        Shaanxi   2015    0.4914 {\VBAR}
 67. {\VBAR}  61       Shandong   2013    0.1372 {\VBAR}
 68. {\VBAR}  62       Shandong   2014    0.1767 {\VBAR}
 69. {\VBAR}  63       Shandong   2015    0.2197 {\VBAR}
 70. {\VBAR}  70       Shanghai   2013   -0.0000 {\VBAR}
 71. {\VBAR}  71       Shanghai   2014   -0.0000 {\VBAR}
 72. {\VBAR}  72       Shanghai   2015   -0.0000 {\VBAR}
 73. {\VBAR}  64         Shanxi   2013   -0.0000 {\VBAR}
 74. {\VBAR}  65         Shanxi   2014   -0.0000 {\VBAR}
 75. {\VBAR}  66         Shanxi   2015    0.0269 {\VBAR}
 76. {\VBAR}  73        Sichuan   2013    0.1667 {\VBAR}
 77. {\VBAR}  74        Sichuan   2014    0.2008 {\VBAR}
 78. {\VBAR}  75        Sichuan   2015    0.2048 {\VBAR}
 79. {\VBAR}  76        Tianjin   2013   -0.0000 {\VBAR}
 80. {\VBAR}  77        Tianjin   2014   -0.0000 {\VBAR}
 81. {\VBAR}  78        Tianjin   2015    0.0116 {\VBAR}
 82. {\VBAR}  79       Xinjiang   2013    0.2433 {\VBAR}
 83. {\VBAR}  80       Xinjiang   2014    0.2511 {\VBAR}
 84. {\VBAR}  81       Xinjiang   2015    0.2397 {\VBAR}
 85. {\VBAR}  82         Yunnan   2013    0.2680 {\VBAR}
 86. {\VBAR}  83         Yunnan   2014    0.3446 {\VBAR}
 87. {\VBAR}  84         Yunnan   2015    0.3416 {\VBAR}
 88. {\VBAR}  85       Zhejiang   2013    0.1197 {\VBAR}
 89. {\VBAR}  86       Zhejiang   2014    0.1540 {\VBAR}
 90. {\VBAR}  87       Zhejiang   2015    0.1732 {\VBAR}
     {\BLC}\HLI{37}{\BRC}
Note: Missing value indicates infeasible problem.
(note: file ex.teddf.result not found)
file ex.teddf.result saved
{\smallskip}
Estimated Results are saved in ex.teddf.result.dta.
{\smallskip}
. 

\end{stlog}

To customize the directional vector, 
\begin{stlog}
	. 
. gen gK=0
{\smallskip}
. gen gL=0
{\smallskip}
. gen gY=Y
{\smallskip}
. gen gCO2=-CO2
{\smallskip}
. teddf K L= Y: CO2, dmu(Province) time(year) gx(gK gL) gy(gY) gb(gCO2) sav(ex.teddf.direction.result,replace)
{\smallskip}
 The directional vector is (gK gL gY gCO2)
{\smallskip}
{\smallskip}
  Directional Distance Function Results:
    (Row: Row \# in the original data; Dv: Estimated value of  DDF.)
{\smallskip}
     {\TLC}\HLI{37}{\TRC}
     {\VBAR} Row       Province   year        Dv {\VBAR}
     {\LFTT}\HLI{37}{\RGTT}
  1. {\VBAR}   1          Anhui   2013    0.4024 {\VBAR}
  2. {\VBAR}   2          Anhui   2014    0.4515 {\VBAR}
  3. {\VBAR}   3          Anhui   2015    0.5049 {\VBAR}
  4. {\VBAR}   4        Beijing   2013   -0.0000 {\VBAR}
  5. {\VBAR}   5        Beijing   2014   -0.0000 {\VBAR}
  6. {\VBAR}   6        Beijing   2015   -0.0000 {\VBAR}
                       ...
                       ...
                       ...
 85. {\VBAR}  82         Yunnan   2013    0.4267 {\VBAR}
 86. {\VBAR}  83         Yunnan   2014    0.4428 {\VBAR}
 87. {\VBAR}  84         Yunnan   2015    0.4648 {\VBAR}
 88. {\VBAR}  85       Zhejiang   2013    0.1507 {\VBAR}
 89. {\VBAR}  86       Zhejiang   2014    0.1891 {\VBAR}
 90. {\VBAR}  87       Zhejiang   2015    0.2265 {\VBAR}
     {\BLC}\HLI{37}{\BRC}
Note: Missing value indicates infeasible problem.
(note: file ex.teddf.direction.result.dta not found)
file ex.teddf.direction.result.dta saved
{\smallskip}
Estimated Results are saved in ex.teddf.direction.result.dta.
{\smallskip}
. 

\end{stlog}

Additionally, we show an application of {\tt teddf} to estimate the non-radial directional distance function model as follows. The Dv variable stores the values of the non-radial directional distance function of the DMUs. B\_K, B\_L, B\_CO2, and B\_Y variables store the reduction proportion of inputs ($K$,$L$) and undesirable outputs ($CO2$), and the expansion proportion of desirable output ($Y$), respectively.

\begin{stlog}
	. 
. teddf K L= Y: CO2, dmu(Province) time(year) nonr sav(ex.teddf.nonr.result,replace)
{\smallskip}
 The weight vector is (1 1 1 1)
{\smallskip}
 The directional vector is (-K -L Y -CO2)
{\smallskip}
{\smallskip}
 Non-raidal Directional Distance Function Results:
    (Row: Row \# in the original data; Dv: Estimated value of Non-raidal DDF.)
{\smallskip}
     {\TLC}\HLI{72}{\TRC}
     {\VBAR} Row       Province   year       Dv      B_K      B_L      B_Y    B_CO2 {\VBAR}
     {\LFTT}\HLI{72}{\RGTT}
  1. {\VBAR}   1          Anhui   2013   1.6710   0.4594   0.7225   0.0000   0.4890 {\VBAR}
  2. {\VBAR}   2          Anhui   2014   1.7823   0.5293   0.7198   0.0000   0.5331 {\VBAR}
  3. {\VBAR}   3          Anhui   2015   1.8210   0.5827   0.7181   0.0000   0.5202 {\VBAR}
  4. {\VBAR}   4        Beijing   2013   0.0000   0.0000   0.0000   0.0000   0.0000 {\VBAR}
  5. {\VBAR}   5        Beijing   2014   0.0000   0.0000   0.0000   0.0000   0.0000 {\VBAR}
  6. {\VBAR}   6        Beijing   2015   0.0000   0.0000   0.0000   0.0000   0.0000 {\VBAR}
                                     ...
                                     ...
                                     ...
 85. {\VBAR}  82         Yunnan   2013   1.7617   0.4480   0.7814   0.0000   0.5323 {\VBAR}
 86. {\VBAR}  83         Yunnan   2014   1.8300   0.5165   0.7834   0.0000   0.5301 {\VBAR}
 87. {\VBAR}  84         Yunnan   2015   1.8184   0.5696   0.7790   0.0000   0.4698 {\VBAR}
 88. {\VBAR}  85       Zhejiang   2013   0.8887   0.2696   0.4386   0.0000   0.1805 {\VBAR}
 89. {\VBAR}  86       Zhejiang   2014   1.0078   0.3364   0.4375   0.0000   0.2340 {\VBAR}
 90. {\VBAR}  87       Zhejiang   2015   1.0589   0.3912   0.4368   0.0000   0.2309 {\VBAR}
     {\BLC}\HLI{72}{\BRC}
Note: Missing value indicates infeasible problem.
(note: file ex.teddf.nonr.result.dta not found)
file ex.teddf.nonr.result.dta saved
{\smallskip}
Estimated Results are saved in ex.teddf.nonr.result.dta.
{\smallskip}
. 

\end{stlog}

To customize the weight matrix, 
\begin{stlog}
	. 
. mat wmatrix=(0.5,0.5,1,1)
{\smallskip}
. teddf K L= Y: CO2, dmu(Province) time(year) nonr wmat(wmatrix) sav(ex.teddf.nonr.weight.result,replace)
{\smallskip}
 The weight vector is (.5 .5 1 1)
{\smallskip}
 The directional vector is (-K -L Y -CO2)
{\smallskip}
{\smallskip}
 Non-raidal Directional Distance Function Results:
    (Row: Row \# in the original data; Dv: Estimated value of Non-raidal DDF.)
{\smallskip}
     {\TLC}\HLI{72}{\TRC}
     {\VBAR} Row       Province   year       Dv      B_K      B_L      B_Y    B_CO2 {\VBAR}
     {\LFTT}\HLI{72}{\RGTT}
  1. {\VBAR}   1          Anhui   2013   1.1480   0.0000   0.4867   0.8499   0.0548 {\VBAR}
  2. {\VBAR}   2          Anhui   2014   1.3351   0.0000   0.4047   1.1247   0.0081 {\VBAR}
  3. {\VBAR}   3          Anhui   2015   1.3577   0.0000   0.3305   1.1924   0.0000 {\VBAR}
  4. {\VBAR}   4        Beijing   2013   0.0000   0.0000   0.0000   0.0000   0.0000 {\VBAR}
  5. {\VBAR}   5        Beijing   2014   0.0000   0.0000   0.0000   0.0000   0.0000 {\VBAR}
  6. {\VBAR}   6        Beijing   2015   0.0000   0.0000   0.0000   0.0000   0.0000 {\VBAR}
  7. {\VBAR}  88      Chongqing   2013   0.7590   0.4994   0.5887   0.0000   0.2149 {\VBAR}
  8. {\VBAR}  89      Chongqing   2014   0.8390   0.5415   0.5781   0.0000   0.2792 {\VBAR}
  9. {\VBAR}  90      Chongqing   2015   0.8217   0.5777   0.5661   0.0000   0.2499 {\VBAR}
 10. {\VBAR}   7         Fujian   2013   0.3997   0.3578   0.4363   0.0000   0.0026 {\VBAR}
 11. {\VBAR}   8         Fujian   2014   0.5734   0.4289   0.4426   0.0000   0.1377 {\VBAR}
 12. {\VBAR}   9         Fujian   2015   0.5298   0.1569   0.1394   0.0000   0.3816 {\VBAR}
 13. {\VBAR}  10          Gansu   2013   1.7086   0.0000   0.5523   1.0853   0.3472 {\VBAR}
 14. {\VBAR}  11          Gansu   2014   1.9726   0.0000   0.4725   1.4444   0.2920 {\VBAR}
 15. {\VBAR}  12          Gansu   2015   2.1542   0.0000   0.3980   1.7971   0.1580 {\VBAR}
 16. {\VBAR}  13      Guangdong   2013   0.0000   0.0000   0.0000   0.0000   0.0000 {\VBAR}
 17. {\VBAR}  14      Guangdong   2014   0.3316   0.1373   0.4425   0.0000   0.0417 {\VBAR}
 18. {\VBAR}  15      Guangdong   2015   0.3449   0.1980   0.4420   0.0000   0.0250 {\VBAR}
 19. {\VBAR}  16        Guangxi   2013   0.9346   0.4916   0.7061   0.0000   0.3357 {\VBAR}
 20. {\VBAR}  17        Guangxi   2014   0.9891   0.5515   0.7041   0.0000   0.3613 {\VBAR}
 21. {\VBAR}  18        Guangxi   2015   0.9813   0.3404   0.5305   0.0000   0.5459 {\VBAR}
 22. {\VBAR}  19        Guizhou   2013   2.0925   0.0000   0.5540   1.3240   0.4915 {\VBAR}
 23. {\VBAR}  20        Guizhou   2014   2.3611   0.0000   0.4772   1.7013   0.4212 {\VBAR}
 24. {\VBAR}  21        Guizhou   2015   2.5863   0.0000   0.3915   2.1054   0.2852 {\VBAR}
 25. {\VBAR}  22         Hainan   2013   0.7708   0.4746   0.6643   0.0000   0.2013 {\VBAR}
 26. {\VBAR}  23         Hainan   2014   0.9317   0.5388   0.6783   0.0000   0.3231 {\VBAR}
 27. {\VBAR}  24         Hainan   2015   1.0204   0.3372   0.4928   0.6054   0.0000 {\VBAR}
 28. {\VBAR}  25          Hebei   2013   1.3533   0.0000   0.2708   0.8446   0.3733 {\VBAR}
 29. {\VBAR}  26          Hebei   2014   1.5132   0.0000   0.1791   1.1007   0.3229 {\VBAR}
 30. {\VBAR}  27          Hebei   2015   1.6128   0.0000   0.0907   1.3424   0.2250 {\VBAR}
 31. {\VBAR}  31   Heilongjiang   2013   0.8150   0.2710   0.4781   0.0000   0.4404 {\VBAR}
 32. {\VBAR}  32   Heilongjiang   2014   0.9176   0.3249   0.4909   0.0000   0.5098 {\VBAR}
 33. {\VBAR}  33   Heilongjiang   2015   0.9388   0.3629   0.4874   0.0000   0.5137 {\VBAR}
 34. {\VBAR}  28          Henan   2013   1.0990   0.0629   0.4881   0.8235   0.0000 {\VBAR}
 35. {\VBAR}  29          Henan   2014   1.2726   0.0737   0.4371   1.0173   0.0000 {\VBAR}
 36. {\VBAR}  30          Henan   2015   1.2915   0.0000   0.3356   1.1237   0.0000 {\VBAR}
 37. {\VBAR}  34          Hubei   2013   0.6580   0.3302   0.5707   0.0000   0.2075 {\VBAR}
 38. {\VBAR}  35          Hubei   2014   0.7519   0.4089   0.5605   0.0000   0.2672 {\VBAR}
 39. {\VBAR}  36          Hubei   2015   0.7501   0.4735   0.5504   0.0000   0.2382 {\VBAR}
 40. {\VBAR}  37          Hunan   2013   0.7284   0.3551   0.6629   0.0000   0.2194 {\VBAR}
 41. {\VBAR}  38          Hunan   2014   0.8019   0.4279   0.6566   0.0000   0.2596 {\VBAR}
 42. {\VBAR}  39          Hunan   2015   0.8527   0.4905   0.6471   0.0000   0.2839 {\VBAR}
 43. {\VBAR}  43        Jiangsu   2013   0.5251   0.2965   0.2874   0.0000   0.2332 {\VBAR}
 44. {\VBAR}  44        Jiangsu   2014   0.5960   0.3588   0.2780   0.0000   0.2776 {\VBAR}
 45. {\VBAR}  45        Jiangsu   2015   0.6134   0.4106   0.2691   0.0000   0.2736 {\VBAR}
 46. {\VBAR}  46        Jiangxi   2013   0.8979   0.4929   0.7009   0.0000   0.3011 {\VBAR}
 47. {\VBAR}  47        Jiangxi   2014   0.9851   0.5492   0.6958   0.0000   0.3626 {\VBAR}
 48. {\VBAR}  48        Jiangxi   2015   1.0345   0.0000   0.2533   0.9078   0.0000 {\VBAR}
 49. {\VBAR}  40          Jilin   2013   1.1570   0.0513   0.0000   1.0484   0.0829 {\VBAR}
 50. {\VBAR}  41          Jilin   2014   1.3032   0.1130   0.0000   1.1092   0.1374 {\VBAR}
 51. {\VBAR}  42          Jilin   2015   1.3102   0.1653   0.0000   1.1775   0.0500 {\VBAR}
 52. {\VBAR}  49       Liaoning   2013   0.9547   0.4781   0.3567   0.0000   0.5373 {\VBAR}
 53. {\VBAR}  50       Liaoning   2014   1.0634   0.2486   0.0000   0.6030   0.3361 {\VBAR}
 54. {\VBAR}  51       Liaoning   2015   1.0673   0.2961   0.0000   0.5701   0.3491 {\VBAR}
 55. {\VBAR}  52      Neimenggu   2013   1.5405   0.1595   0.0000   0.7735   0.6873 {\VBAR}
 56. {\VBAR}  53      Neimenggu   2014   1.6818   0.2282   0.0000   0.8613   0.7064 {\VBAR}
 57. {\VBAR}  54      Neimenggu   2015   1.6696   0.2697   0.0000   0.8268   0.7080 {\VBAR}
 58. {\VBAR}  55        Ningxia   2013   3.0587   0.0000   0.1265   2.3015   0.6939 {\VBAR}
 59. {\VBAR}  56        Ningxia   2014   3.4791   0.0000   0.0072   2.7891   0.6864 {\VBAR}
 60. {\VBAR}  57        Ningxia   2015   3.5787   0.0954   0.0000   2.8435   0.6875 {\VBAR}
 61. {\VBAR}  58        Qinghai   2013   1.7290   0.0000   0.1554   1.6513   0.0000 {\VBAR}
 62. {\VBAR}  59        Qinghai   2014   1.9572   0.0977   0.1108   1.8529   0.0000 {\VBAR}
 63. {\VBAR}  60        Qinghai   2015   2.1083   0.0971   0.0000   2.0598   0.0000 {\VBAR}
 64. {\VBAR}  67        Shaanxi   2013   1.7282   0.0000   0.1839   1.4778   0.1585 {\VBAR}
 65. {\VBAR}  68        Shaanxi   2014   1.9543   0.0000   0.0650   1.7879   0.1340 {\VBAR}
 66. {\VBAR}  69        Shaanxi   2015   2.0558   0.0399   0.0000   1.9710   0.0648 {\VBAR}
 67. {\VBAR}  61       Shandong   2013   0.8202   0.2871   0.4964   0.0000   0.4285 {\VBAR}
 68. {\VBAR}  62       Shandong   2014   0.9095   0.3519   0.4916   0.0000   0.4878 {\VBAR}
 69. {\VBAR}  63       Shandong   2015   0.9582   0.4106   0.4904   0.0000   0.5077 {\VBAR}
 70. {\VBAR}  70       Shanghai   2013   0.0000   0.0000   0.0000   0.0000   0.0000 {\VBAR}
 71. {\VBAR}  71       Shanghai   2014   0.0000   0.0000   0.0000   0.0000   0.0000 {\VBAR}
 72. {\VBAR}  72       Shanghai   2015   0.0000   0.0000   0.0000   0.0000   0.0000 {\VBAR}
 73. {\VBAR}  64         Shanxi   2013   1.9164   0.0000   0.2748   1.0972   0.6818 {\VBAR}
 74. {\VBAR}  65         Shanxi   2014   2.2083   0.0000   0.1839   1.4368   0.6795 {\VBAR}
 75. {\VBAR}  66         Shanxi   2015   2.5003   0.0000   0.0894   1.8447   0.6109 {\VBAR}
 76. {\VBAR}  73        Sichuan   2013   0.7386   0.4147   0.6779   0.0000   0.1923 {\VBAR}
 77. {\VBAR}  74        Sichuan   2014   0.8285   0.4715   0.6752   0.0000   0.2551 {\VBAR}
 78. {\VBAR}  75        Sichuan   2015   0.8076   0.1993   0.4827   0.0000   0.4666 {\VBAR}
 79. {\VBAR}  76        Tianjin   2013   0.4554   0.3760   0.0466   0.0000   0.2441 {\VBAR}
 80. {\VBAR}  77        Tianjin   2014   0.5129   0.4253   0.0551   0.0000   0.2727 {\VBAR}
 81. {\VBAR}  78        Tianjin   2015   0.4726   0.4665   0.0581   0.0000   0.2103 {\VBAR}
 82. {\VBAR}  79       Xinjiang   2013   1.9188   0.0000   0.1794   1.2920   0.5370 {\VBAR}
 83. {\VBAR}  80       Xinjiang   2014   2.2027   0.0000   0.0731   1.6136   0.5526 {\VBAR}
 84. {\VBAR}  81       Xinjiang   2015   2.4641   0.0096   0.0000   1.9262   0.5330 {\VBAR}
 85. {\VBAR}  82         Yunnan   2013   1.2663   0.0000   0.6040   0.8117   0.1526 {\VBAR}
 86. {\VBAR}  83         Yunnan   2014   1.3723   0.0000   0.5520   1.0682   0.0281 {\VBAR}
 87. {\VBAR}  84         Yunnan   2015   1.2843   0.0000   0.4933   1.0377   0.0000 {\VBAR}
 88. {\VBAR}  85       Zhejiang   2013   0.5346   0.2696   0.4386   0.0000   0.1805 {\VBAR}
 89. {\VBAR}  86       Zhejiang   2014   0.6209   0.3364   0.4375   0.0000   0.2340 {\VBAR}
 90. {\VBAR}  87       Zhejiang   2015   0.6449   0.3912   0.4368   0.0000   0.2309 {\VBAR}
     {\BLC}\HLI{72}{\BRC}
Note: Missing value indicates infeasible problem.
(note: file ex.teddf.nonr.weight.result not found)
file ex.teddf.nonr.weight.result saved
{\smallskip}
Estimated Results are saved in ex.teddf.nonr.weight.result.dta.
{\smallskip}
. 

\end{stlog}


\subsection{Application of \textit{gtfpch}}
We first apply {\tt gtfpch} to estimate the Malmquist–Luenberger productivity index (MLPI) to measure the green total-factor productivity growth of China's provinces. Regarding the results, TFPCH stores the values of MLPI; TECH and TECCH are the two decomposition terms of MLPI, describing technical efficiency change and technological change, respectively. Note that we implement the estimation based on the global technology benchmark by specifying the \textit{global} option.

\begin{stlog}
	. 
. egen id=group(Province)
{\smallskip}
. xtset id year
       panel variable:  id (strongly balanced)
        time variable:  year, 2013 to 2015
                delta:  1 unit
{\smallskip}
. gtfpch K L= Y: CO2, dmu(Province) global sav(ex.gtfpch.result,replace)
{\smallskip}
 The directional vector is (0 0 Y -CO2)
{\smallskip}
{\smallskip}
 Total Factor Productivity Change:Malmquist-Luenberger Productivity Index
    (Row: Row \# in the original data; Pdwise: periodwise)
{\smallskip}
     {\TLC}\HLI{64}{\TRC}
     {\VBAR} Row       Province   id      Pdwise    TFPCH     TECH    TECCH {\VBAR}
     {\LFTT}\HLI{64}{\RGTT}
  1. {\VBAR}   2          Anhui    1   2013{\tytilde}2014   0.9943   0.9662   1.0290 {\VBAR}
  2. {\VBAR}   3          Anhui    1   2014{\tytilde}2015   0.9951   0.9645   1.0317 {\VBAR}
  3. {\VBAR}   5        Beijing    2   2013{\tytilde}2014   1.0328   1.0000   1.0328 {\VBAR}
  4. {\VBAR}   6        Beijing    2   2014{\tytilde}2015   1.0583   1.0000   1.0583 {\VBAR}
  5. {\VBAR}   8      Chongqing    3   2013{\tytilde}2014   1.0013   0.9883   1.0132 {\VBAR}
  6. {\VBAR}   9      Chongqing    3   2014{\tytilde}2015   1.0222   0.9659   1.0582 {\VBAR}
  7. {\VBAR}  11         Fujian    4   2013{\tytilde}2014   0.9813   0.9438   1.0397 {\VBAR}
  8. {\VBAR}  12         Fujian    4   2014{\tytilde}2015   1.0207   0.9774   1.0443 {\VBAR}
  9. {\VBAR}  14          Gansu    5   2013{\tytilde}2014   0.9942   0.9748   1.0200 {\VBAR}
 10. {\VBAR}  15          Gansu    5   2014{\tytilde}2015   0.9958   0.9725   1.0240 {\VBAR}
 11. {\VBAR}  17      Guangdong    6   2013{\tytilde}2014   1.0185   0.9575   1.0637 {\VBAR}
 12. {\VBAR}  18      Guangdong    6   2014{\tytilde}2015   1.0152   0.9841   1.0316 {\VBAR}
 13. {\VBAR}  20        Guangxi    7   2013{\tytilde}2014   1.0031   0.9856   1.0177 {\VBAR}
 14. {\VBAR}  21        Guangxi    7   2014{\tytilde}2015   1.0317   0.9830   1.0495 {\VBAR}
 15. {\VBAR}  23        Guizhou    8   2013{\tytilde}2014   1.0014   0.9889   1.0127 {\VBAR}
 16. {\VBAR}  24        Guizhou    8   2014{\tytilde}2015   1.0080   0.9879   1.0204 {\VBAR}
 17. {\VBAR}  26         Hainan    9   2013{\tytilde}2014   0.9729   0.9497   1.0244 {\VBAR}
 18. {\VBAR}  27         Hainan    9   2014{\tytilde}2015   0.9773   0.9321   1.0485 {\VBAR}
 19. {\VBAR}  29          Hebei   10   2013{\tytilde}2014   1.0052   0.9796   1.0261 {\VBAR}
 20. {\VBAR}  30          Hebei   10   2014{\tytilde}2015   1.0002   0.9784   1.0223 {\VBAR}
 21. {\VBAR}  32   Heilongjiang   11   2013{\tytilde}2014   1.0007   0.9546   1.0482 {\VBAR}
 22. {\VBAR}  33   Heilongjiang   11   2014{\tytilde}2015   1.0071   0.9854   1.0220 {\VBAR}
 23. {\VBAR}  35          Henan   12   2013{\tytilde}2014   0.9955   0.9689   1.0274 {\VBAR}
 24. {\VBAR}  36          Henan   12   2014{\tytilde}2015   0.9963   0.9632   1.0343 {\VBAR}
 25. {\VBAR}  38          Hubei   13   2013{\tytilde}2014   1.0019   0.9618   1.0418 {\VBAR}
 26. {\VBAR}  39          Hubei   13   2014{\tytilde}2015   1.0015   0.9656   1.0372 {\VBAR}
 27. {\VBAR}  41          Hunan   14   2013{\tytilde}2014   1.0090   0.9714   1.0388 {\VBAR}
 28. {\VBAR}  42          Hunan   14   2014{\tytilde}2015   0.9820   0.9453   1.0388 {\VBAR}
 29. {\VBAR}  44        Jiangsu   15   2013{\tytilde}2014   1.0275   0.9894   1.0385 {\VBAR}
 30. {\VBAR}  45        Jiangsu   15   2014{\tytilde}2015   1.0390   0.9975   1.0417 {\VBAR}
 31. {\VBAR}  47        Jiangxi   16   2013{\tytilde}2014   0.9932   0.9731   1.0206 {\VBAR}
 32. {\VBAR}  48        Jiangxi   16   2014{\tytilde}2015   0.9970   0.9515   1.0478 {\VBAR}
 33. {\VBAR}  50          Jilin   17   2013{\tytilde}2014   1.0064   0.9858   1.0209 {\VBAR}
 34. {\VBAR}  51          Jilin   17   2014{\tytilde}2015   1.0312   0.9922   1.0393 {\VBAR}
 35. {\VBAR}  53       Liaoning   18   2013{\tytilde}2014   1.0179   0.9712   1.0481 {\VBAR}
 36. {\VBAR}  54       Liaoning   18   2014{\tytilde}2015   1.0289   0.9997   1.0292 {\VBAR}
 37. {\VBAR}  56      Neimenggu   19   2013{\tytilde}2014   1.0074   0.9830   1.0249 {\VBAR}
 38. {\VBAR}  57      Neimenggu   19   2014{\tytilde}2015   1.0157   1.0019   1.0138 {\VBAR}
 39. {\VBAR}  59        Ningxia   20   2013{\tytilde}2014   1.0024   0.9966   1.0058 {\VBAR}
 40. {\VBAR}  60        Ningxia   20   2014{\tytilde}2015   1.0021   0.9914   1.0108 {\VBAR}
 41. {\VBAR}  62        Qinghai   21   2013{\tytilde}2014   1.0127   0.9928   1.0200 {\VBAR}
 42. {\VBAR}  63        Qinghai   21   2014{\tytilde}2015   1.0038   0.9657   1.0395 {\VBAR}
 43. {\VBAR}  65        Shaanxi   22   2013{\tytilde}2014   0.9983   0.9908   1.0076 {\VBAR}
 44. {\VBAR}  66        Shaanxi   22   2014{\tytilde}2015   1.0120   0.9772   1.0357 {\VBAR}
 45. {\VBAR}  68       Shandong   23   2013{\tytilde}2014   1.0032   0.9588   1.0463 {\VBAR}
 46. {\VBAR}  69       Shandong   23   2014{\tytilde}2015   0.9919   0.9685   1.0241 {\VBAR}
 47. {\VBAR}  71       Shanghai   24   2013{\tytilde}2014   0.9955   1.0000   0.9955 {\VBAR}
 48. {\VBAR}  72       Shanghai   24   2014{\tytilde}2015   1.0152   1.0000   1.0152 {\VBAR}
 49. {\VBAR}  74         Shanxi   25   2013{\tytilde}2014   0.9932   0.9788   1.0147 {\VBAR}
 50. {\VBAR}  75         Shanxi   25   2014{\tytilde}2015   0.9967   0.9855   1.0114 {\VBAR}
 51. {\VBAR}  77        Sichuan   26   2013{\tytilde}2014   1.0027   0.9699   1.0338 {\VBAR}
 52. {\VBAR}  78        Sichuan   26   2014{\tytilde}2015   1.0187   0.9774   1.0423 {\VBAR}
 53. {\VBAR}  80        Tianjin   27   2013{\tytilde}2014   1.0686   1.0000   1.0686 {\VBAR}
 54. {\VBAR}  81        Tianjin   27   2014{\tytilde}2015   1.0697   0.9701   1.1027 {\VBAR}
 55. {\VBAR}  83       Xinjiang   28   2013{\tytilde}2014   0.9897   0.9746   1.0154 {\VBAR}
 56. {\VBAR}  84       Xinjiang   28   2014{\tytilde}2015   0.9881   0.9727   1.0159 {\VBAR}
 57. {\VBAR}  86         Yunnan   29   2013{\tytilde}2014   1.0161   0.9889   1.0275 {\VBAR}
 58. {\VBAR}  87         Yunnan   29   2014{\tytilde}2015   1.0155   0.9849   1.0310 {\VBAR}
 59. {\VBAR}  89       Zhejiang   30   2013{\tytilde}2014   1.0143   0.9677   1.0481 {\VBAR}
 60. {\VBAR}  90       Zhejiang   30   2014{\tytilde}2015   1.0028   0.9695   1.0343 {\VBAR}
     {\BLC}\HLI{64}{\BRC}
Note: missing value indicates infeasible problem.
(note: file ex.gtfpch.result not found)
file ex.gtfpch.result saved
{\smallskip}
Estimated Results are saved in ex.gtfpch.result.dta.
{\smallskip}
. 

\end{stlog}

Alternatively, {\tt gtfpch} can be employed to estimate the Luenberger productivity indicator. We present an example as follows.  

\begin{stlog}
	. 
. gtfpch K L= Y: CO2, dmu( Province ) nonr  global sav(ex.gtfpch.nonr.result,replace)
{\smallskip}
 The weight vector is (0 0 1 1)
{\smallskip}
 The directional vector is (0 0 Y -CO2)
{\smallskip}
{\smallskip}
 Total Factor Productivity Change:Luenberger Productivity Index (based on nonradial DDF)
    (Row: Row \# in the original data; Pdwise: periodwise)
{\smallskip}
     {\TLC}\HLI{66}{\TRC}
     {\VBAR} Row       Province   id      Pdwise     TFPCH      TECH    TECCH {\VBAR}
     {\LFTT}\HLI{66}{\RGTT}
  1. {\VBAR}   2          Anhui    1   2013{\tytilde}2014   -0.0676   -0.2281   0.1605 {\VBAR}
  2. {\VBAR}   3          Anhui    1   2014{\tytilde}2015    0.0214   -0.0597   0.0811 {\VBAR}
  3. {\VBAR}   5        Beijing    2   2013{\tytilde}2014    0.0832   -0.0000   0.0832 {\VBAR}
  4. {\VBAR}   6        Beijing    2   2014{\tytilde}2015    0.1705    0.0000   0.1705 {\VBAR}
  5. {\VBAR}   8      Chongqing    3   2013{\tytilde}2014    0.0175   -0.0564   0.0738 {\VBAR}
  6. {\VBAR}   9      Chongqing    3   2014{\tytilde}2015    0.0178   -0.1079   0.1257 {\VBAR}
                                     ...
                                     ...
                                     ...
 55. {\VBAR}  83       Xinjiang   28   2013{\tytilde}2014   -0.2221   -0.3371   0.1150 {\VBAR}
 56. {\VBAR}  84       Xinjiang   28   2014{\tytilde}2015   -0.2232   -0.2931   0.0699 {\VBAR}
 57. {\VBAR}  86         Yunnan   29   2013{\tytilde}2014    0.0128   -0.1320   0.1448 {\VBAR}
 58. {\VBAR}  87         Yunnan   29   2014{\tytilde}2015    0.1378    0.0586   0.0792 {\VBAR}
 59. {\VBAR}  89       Zhejiang   30   2013{\tytilde}2014    0.0119   -0.0558   0.0677 {\VBAR}
 60. {\VBAR}  90       Zhejiang   30   2014{\tytilde}2015   -0.0092   -0.0996   0.0903 {\VBAR}
     {\BLC}\HLI{66}{\BRC}
Note: missing value indicates infeasible problem.
(note: file ex.gtfpch.nonr.result.dta not found)
file ex.gtfpch.nonr.result.dta saved
{\smallskip}
Estimated Results are saved in ex.gtfpch.nonr.result.dta.
{\smallskip}
. 

\end{stlog}



\section{Conclusion}\label{sec_conclusion}
With the increasing demand for improving sustainability at the macro and micro levels, scholars and managers recognized that it is becoming more and more important to consider undesirable output in efficiency and productivity analysis. 
Stata, as one of the leading packages for economic analysis, however, has not provided comprehensive tools to measure technical efficiency and total factor productivity change when considering undesirable outputs. 
Here, as an attempt to fill this gap, we introduced two new Stata commands that perform estimations for nonparametric frontier models with undesirable outputs.

{\tt teddf} estimates directional distance function with undesirable outputs for technical efficiency measurement. Both radial Debreu-Farrell and non-radial Russell measures can be calculated, under different assumptions about the production technology, e.g., window, biennial, sequential, and global production technology.
{\tt gtfpch} measures total factor productivity change with undesirable outputs using Malmquist–Luenberger productivity index or Luenberger indicator. Two types of specifications of decomposing total factor productivity change were given. 
Some empirical examples have been presented to show the usage of the two commands.

Finally, it should be noted that the models we introduced are DEA-type estimators, which might be sensitive to sampling variation. Thus, the statistical inference in this context is critical. However, there are still many open issues both in theory and application.



\section{Acknowledgments}
Kerui Du thanks the financial support of the National Natural Science Foundation of China (72074184) and the Fundamental Research Funds for the Central Universities (20720201016). Ning Zhang thanks the financial support of the National Natural Science Foundation of China (72033005; 71822402). We are grateful to Stephen P. Jenkins and the anonymous reviewer for the helpful comments and suggestions which led to an improved version of this paper.


\endinput
